\documentclass[a4paper,12pt]{article}
\usepackage[left=25mm, right=15mm, top=20mm, bottom=20mm, footskip=10mm]{geometry}
\usepackage[T2A]{fontenc}   % Руссификация
\usepackage{inputenc}       % UTF-8
\usepackage[russian]{babel} % 
\usepackage{amsmath,amssymb}% Специальные символы
\usepackage{indentfirst}    % Отступ первой строки абзаца
\usepackage{ragged2e}       % Выравнивание по ширине
%\usepackage{newtxtext,newtxmath} % Приближение шрифта к Times New Roman

\title{Проект «Patria»}
\author{}
\date{}

% Устанавливаем название оглавления
\renewcommand*\contentsname{Оглавление}

% Для создания подчеркиваний (использование: "\hrf{1cm}")
\def\hrf#1{\hbox to#1{\hrulefill}}

% Нумеруем списке так "1. \\ 1.1 \\ 1.2 \\ 2. \\ 2.1 ...
\renewcommand{\labelenumii}{\arabic{enumi}.\arabic{enumii}.}

% Шрифт документа "sans-serif"
\renewcommand{\familydefault}{\sfdefault}


\begin{document}
% ---------------------------------------------------
% Титульный лист
\begin{titlepage}

    \begin{center}
        Министерство науки и высшего образования Российской Федерации \\
        Федеральное государственное бюджетное образовательное учреждение \\
        высшего образования \\
        «Алтайский государственный технический университет им. И. И. Ползунова»
    \end{center}
    \vskip 1cm 
    Факультет информационных технологий \\
    Кафедра прикладной математики
    
    \vskip 2cm 
    
    \begin{flushright}
        \parbox{8cm}{
            Отчет защищен с оценкой \hrulefill  \\
            Преподаватель \hrulefill (подпись)  \\
            \centerline{«\hrf{0.7cm}» \hrf{2.2cm} 2025 г.}
        }
    \end{flushright}
    
    \vskip 3cm 
    
    \begin{center}
        Отчет \\
        по курсовой работе \\
        дисциплина «Операционные системы»
    \end{center}
    
    \vskip 4cm 
    
    \begin{flushright}
        Студент гр. ПИ-21 \\
        Горбунов В.Е.
        
        Преподаватель, к.т.н., доцент, \\
        Боровцов Е. Г.        
    \end{flushright}
    
    \vskip 4cm 
    
    \centerline{Барнаул 2025}
    
    \end{titlepage}

% Оглавление
\tableofcontents

% Текст документа
\maketitle

% ---------------------------------------------------
\section{Описание проекта}
	Программный комплекс «Patria» является свободно распростроняемым и открытым программным продуктом и может быть использован или модифицирован согласно настоящей лицензии. «Patria» выполняет функции «мессенджера» - сетевого обмена данными между двумя или одним и множеством участников диалога.

    Цель проекта. Создать уникальное ПО, обеспечивающее бесперебойную (устойчивость сервера к высоким нагрузком до 10.000 подключений), качественную (достоверная передача информации с использованием Unicode кодировки), защищенную (алгоритмы шифрования не позволяют получить информацию в разумные сроки без ключа) связь между корреспондентами
	
    Программный комплекс «Patria» состоит из двух независимых программ:
\begin{itemize}
    \item [$\triangleright$] Серверное ПО
    \item [$\triangleright$] Клиентское ПО
\end{itemize}

% ---------------------------------------------------
\section{Серверное программное обеспечение}
\subsection{Описание}
	Серверное ПО - программная часть продукта, которая обеспечивает связь между клиентами, хранит и индексирует сообщения в зашифрованном виде, а также предоставляет сообщения пользователям, имеющим соответствующий доступ.

\subsection{Функции}
\begin{enumerate}
    \item Обработка сообщений \\    
    Сервер использует файловую систему в качестве простейшей базы данных, которая отвечает за часть с хранением, индексацией и выдачей сообщений клиентам. Вся информация хранится в зашифрованном виде.

    \item Сетевое соединение \\
	Для осуществления функционала сетевого менеджмента, сервер использует функции "C". Многопоточная обработка клиентов реализуется функционалом языка, т.е. никаких зависимостей от внешних библиотек
\end{enumerate}

\subsection{Техническое обеспечение}
	В роли сервера может выступать любая ЭВМ, имеющая постоянный накопитель размером не менее 100 Гб, сетевой интерфейс, процессор Intel Pentium и более старшие поколения (или их аналог), объем оперативной памяти не менее 300 Мб, сервер может управляться с помощью удаленного доступа из терминала по ssh подключению на 22 порт.

\subsection{Зависимости}
    Исключительное использование библиотек C и C++

\subsection{Архитектура}


\subsection{Описание сущностей и функций}


% ---------------------------------------------------
\section{Клиентское программное обеспечение}
\subsection{Описание}
    Клиентское ПО - программная часть, предоставляющая пользователю графический интерфейс по управлению функционалом мессенджера. Является программой для оконечного устройства сети, обеспечивающего коммуникацию с другими устройствами. 

\subsection{Функции}
\begin{enumerate}
    \item Графический интерфейс
    \item Настройки конфиденциальности
    \item Управление аккаунтом
    \item Создание чатов с одним или множеством участников
    \item Управление историей сообщений (удаление, редактирование)
    \item Использование реакций
    \item Передача различных файлов, в том числе изображений
\end{enumerate}

\subsection{Техническое обеспечение}
    В роли оконечного устройства может выступать любая ЭВМ, имеющая постоянный накопитель размером не менее 1 Гб (для установки программи и хэширования), сетевой интерфейс, процессор Intel Pentium и более старшие поколения (или их аналоги)

\subsection{Зависимости}
\begin{enumerate}
    \item SFML \\
    Обеспечивает GUI (Graphical User Interface).
\end{enumerate}

\subsection{Архитектура}


\subsection{Описание сущностей и функций}


\section{Обзор ПО}
\subsection{Windows}
    Тестирование работы сервера на Windows
    
    Тестирование работы клиента на Windows 
    

\subsection{GNU/Linux}
    Тестирование работы сервера на GNU/Linux
    
    Тестирование работы клиента на GNU/Linux
    
\end{document}